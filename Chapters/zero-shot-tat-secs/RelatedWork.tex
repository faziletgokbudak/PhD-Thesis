\section{Related Work}\label{zero-shot-RW}
This work fall into the broader category of image-to-image translation. Here, I will review the recent works on diffusion-based techniques.

\subsection{Text-to-Image Synthesis.}

Recent years have seen significant advancements in text-to-image synthesis from the initial Generative Adversarial Network (GAN)-based approaches \cite{li2020manigan,xu2018attngan, zhang2017stackgan, zhang2018stackgan++} to the latest diffusion \cite{gu2022vector, ho2020denoising,nichol2021improved,ramesh2022hierarchical, rombach2022high, saharia2022photorealistic,song2020denoising,zhang2023adding} and transformer models \cite{ding2022cogview2, esser2021taming, ramesh2021zero, yu2022scaling}. Notably, DALL·E 2 \cite{ramesh2022hierarchical} and Imagen \cite{ho2022imagen} apply diffusion models guided by text prompts. 



 Other approaches [\cite{crowson2022vqgan, ramesh2022hierarchical}] leverage pre-trained CLIP models \cite{radford2021learning} to guide image generation based on textual descriptions. 
 
 More recently, Stable Diffusion \cite{rombach2022high}, trained on large image-text pairs \cite{schuhmann2021laion}, has been made publicly available and has served as the foundation for numerous image generation and manipulation works. ControlNet \cite{zhang2023adding} proposes to control Stable Diffusion with spatially localized conditions for image synthesis. Different from these works, we aim to introduce a component that can enable pre-trained text-to-image models for mask-free local image editing.

\subsection{Text-guided image manipulation}
Several recent works
have utilized pre-trained generator models and CLIP \cite{radford2021learning} 

for text-driven image manipulation \cite{bar2022text2live, gal2022stylegan,kwon2022clipstyler, liu2021fusedream, patashnik2021styleclip} [2, 3, 17, 30, 35, 41].
StyleCLIP \cite{patashnik2021styleclip} combines the generative ability of StyleGAN \cite{karras2020analyzing} with CLIP to control latent codes, enabling a
wide range of image manipulations. VQGAN-CLIP \cite{crowson2022vqgan}
uses CLIP \cite{radford2021learning} to guide VQ-GAN \cite{esser2021taming} for high-quality image generation and editing.

There are several approaches [1, 18, 26, 27, 36, 37, 39,
47, 53] \cite{avrahami2022blended, gal2022image, kawar2023imagic, kim2022diffusionclip, liu2023more, meng2021sdedit,nichol2021glide,ruiz2023dreambooth}

that use diffusion models for text-driven image manipulation. Imagic \cite{kawar2023imagic} can generate textual embeddings
aligning with the input images and editing prompts, and
fine-tune the diffusion model to perform edits. InstructPix2Pix \cite{brooks2023instructpix2pix} combines GPT-3 \cite{brown2020language} and Stable Diffusion \cite{rombach2022high}
to edit images with human instructions. Our work is related to the state-of-the-art methods DiffEdit \cite{couairon2022diffedit} and MasaCtrl \cite{cao2023masactrl}. DiffEdit  \cite{couairon2022diffedit}  leverages DDIM inversion \cite{dhariwal2021diffusion, song2020denoising}with
the automatically produced masks for local image editing.
MasaCtrl \cite{cao2023masactrl} proposes mutual self-attention and learns the
editing masks from the cross-attention maps of the diffusion
models. Motivated by the aforementioned works, we also
utilize diffusion models and CLIP guidance for text-driven
image manipulation. In contrast to DiffEdit  \cite{couairon2022diffedit}  and MasaCtrl \cite{cao2023masactrl}, whose editing is much more sensitive to the generated mask regions, our proposed method focuses on learning bounding boxes for local editing, which can be more
flexible in accommodating diverse text prompts.