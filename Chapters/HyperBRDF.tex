\chapter{HyperBRDF: Neural Generalizable Material Representation}
\label{ch:HyperBRDF}

Appearance is the byproduct of illumination and reflectance, in which the two interact with each other, covering the shape of the objects. Photographs can only capture the appearance with scene components coupled together. Previous chapters discuss appearance manipulations at the image level without any explicit effort for the decomposition of human-distinguishable scene properties. Image-based editing techniques are instrumental in many real-world applications; however, they suffer from the lack of explicit control over scene properties, such as light sources, material types or object geometries. This chapter goes beyond the image-level editing of appearance and brings a physics and graphics aspect to appearance representations. More specifically, I propose a neural representation of reflectance, which can be incorporated into renderers for photorealistic simulation of the scene appearance.

Bidirectional reflectance distribution functions (\gls{BRDF}) define the reflectance as a function of illumination and viewing directions. This chapter presents HyperBRDF, a technique to estimate measured \gls{BRDF}s from a sparse set of samples. HyperBRDF offers accurate \gls{BRDF} reconstructions that are generalisable to new materials. This opens the door to \gls{BRDF} reconstructions from a variety of data sources. The success of the approach relies on the ability of hypernetworks to generate a robust representation of \gls{BRDF}s and a set encoder that allows us to feed inputs of different sizes to the architecture. The set encoder and the hypernetwork also enable the compression of densely sampled \gls{BRDF}s. I evaluate the technique both qualitatively and quantitatively on the well-known MERL dataset of 100 isotropic materials. HyperBRDF accurately 1) estimates the \gls{BRDF}s of unseen materials even for an extremely sparse sampling, 2) compresses the measured \gls{BRDF}s into very small embeddings, e.g., 7D. 


\section{Introduction}
\label{sec:intro}


In computer graphics and vision, achieving realistic renderings for intricate surface materials hinges on accurately describing the interaction of light with the surfaces. This is conventionally conveyed through the modelling and reconstruction of a 4D Bidirectional Reflectance Distribution Function (BRDF), which quantifies the relationship between the incident and outgoing light intensities for a specific material. In this chapter, I propose a novel generalisable \gls{BRDF} representation model that can estimate the \gls{BRDF}s of new materials from highly sparse and unstructured point-based samples \footnote{Point-based samples refer to the \gls{BRDF} values acquired at specific points on a surface with known viewing and lighting directions.} and compress the densely sampled values into very small latent embeddings.


While there has been prior work that attempts to tackle similar tasks by estimating the free parameters of analytic \gls{BRDF}s or the principal components of \gls{BRDF}s from images or reflectance measurements, the oversimplified models of the complex \gls{BRDF} function lead to inaccuracies when predicting real materials, thereby diminishing the realism in renderings~\cite{ngan2005}. Moreover, the process of fitting through nonlinear optimisation is inherently unstable, computationally expensive, and prone to local minima, hindering the accurate reconstruction of material appearance~\cite{dupuy2015, guarnera2016}. Measured \gls{BRDF}s of real-world materials can also be fraught with errors due to equipment limitations, adding to the complexity of the fitting process~\cite{nielsen2015optimal}. 

\begin{figure}
  \centering
   \includegraphics[width=\linewidth]{Chapters/hyperbrdf-figs/teaser_cropped.png}
   \caption{A room scene rendered with materials reconstructed by HyperBRDF, including sparse reconstruction (table top and legs, door, door and picture frames, hinge), compression (two teapots on the left, door handle) and \gls{BRDF} interpolation (right-most teapot). Scene courtesy of [Benedikt Bitterli].}
   \label{fig:hyper-teaser}
\end{figure}

Recent advances in deep learning have enabled the accurate representation of complex continuous signals (\textit{e.g.} images, surfaces, volumes, materials, \textit{etc.}) using a Neural Field~\cite{sitzmann2020siren, ffn, cnf2023}, \textit{i.e.} a neural network mapping coordinate inputs to sampled values, without compromising model compactness. Consequently, neural fields have gained substantial popularity for representing continuous \gls{BRDF}s in recent works~\cite{sztrajman2021neural, cnf2023}. However, reconstructing a neural field representation for \gls{BRDF}s typically requires training the network with a regression loss function to overfit to a single material, which demands dense sampling and extensive computational resources, while being unable to generalise to new materials.

More recently, Generalisable Neural Fields~\cite{rebain2022attention} (\gls{GNF}s) have emerged as a promising solution to the aforementioned challenges. Rather than overfitting to individual signals, \gls{GNF}s are designed to learn a generalised mapping, either deterministic or probabilistic, between sampled signals and their full neural field representations in a fashion similar to supervised learning. The key strategy of \gls{GNF}s involves conditioning the neural fields with a latent embedding of the signal samples. Popular conditioning mechanisms include concatenated latent vectors~\cite{park2019deepsdf}, hypernetworks~\cite{ha2017hypernetworks}, and attention-based set latent representations~\cite{jiang2021cotr}. Once the conditional neural field is properly learned, reconstruction of the full signal can be achieved at inference time, and even with highly sparse and unstructured samples.


Inspired by \gls{GNF}s, I propose a novel framework, HyperBRDF, for generalisable neural \gls{BRDF} representation for both the estimation of unseen materials from sparse and unstructured samples and the compression of measured \gls{BRDF}s into low-dimensional latent space. In this framework, I employ a multi-layer perceptron (\gls{MLP}) model as the neural field backbone for \gls{BRDF} representation, a hyper-network for conditioning, and a set encoder that allows for mapping an arbitrary number of reflectance measurements from arbitrary directions to a compact latent embedding for conditional \gls{BRDF} generation. 
The built-in nonlinear interpolation capability of the hyper-network also offers robust and adaptable material editing and blending across various sample sizes.


Unlike previous work, the \gls{BRDF} reconstruction by HyperBRDF is highly efficient without the need for extensive training to overfit individual materials, while also maintaining state-of-the-art performance in reconstruction accuracy for sample size below 4000; see Figure~\ref{fig:imp_comp_upt}.


In summary, HyperBRDF offers a novel solution for the challenging task of generalisable \gls{BRDF} modelling with the following key contributions:
\begin{itemize}
    \item{\textbf{Generalisability and Adaptability:} HyperBRDF ensures robustness and adaptability across varying sample sizes and appearances, making it highly effective for estimating the \gls{BRDF}s of unseen materials from highly sparse and unstructured sampling. Its adaptability also extends to highly compact representation of \gls{BRDF}s, overall outperforming the prior state-of-the-art compression method; see Table \ref{table: oursvsnps}.}
    

    \item{\textbf{Realism and Accuracy:}
    My extensive evaluation demonstrates the superior performance of HyperBRDF in reconstructing the \gls{BRDF}s of 20 test materials from a limited number of samples, ranging from 40 to 4000, outperforming prior methods in terms of appearance modelling and colour preservation (by at least 2dB in peak-signal-to-noise ratio and 1 in Delta E).
    }
\end{itemize}

\section{Motivation and impact}
\label{sec:hyperbrdf-mot}

The representation of measured \gls{BRDF}s is crucial in computer graphics, especially in physically-based rendering (\gls{PBR}) for applications such as filmmaking, gaming, virtual and augmented reality, and product design, aiming for immersive realism. \gls{PBR} provides a more precise portrayal of light-material interactions. However, real-world \gls{BRDF} models are reconstructed by taking samples with a device, such as a camera or gonioreflectometer and often demand numerous samples, making acquisition, storage, and manipulation costly and challenging. For instance, the MERL dataset \cite{Matusik2003jul}, the main dataset I utilised for experiments, captures 330 high dynamic range images with a CCD camera for each material in roughly four hours. The RGL dataset \cite{dupuy2018adaptive}, the secondary dataset included in training, captures samples with a gonioreflectometer that leads to captures times of approximately 2.5 hours for isotropic materials. Therefore, HyperBRDF offers multiple impactful contributions: 
\begin{itemize}
 \item can integrate with professional setups where capturing only one material can take hours. HyperBRDF’s superior sparse reconstruction ability can help significantly reduce the manual and heavy workload of material acquisition. I also discuss the related work aligned with this research line in Section \ref{hyperbrdf-RW}.
 \item Providing \gls{BRDF} compression with high accuracy and efficiency, encoding over a million samples into a 7D latent vector, saving storage space and increasing transfer speed. 
 \item Allowing easy material editing through its latent space representation.
 \end{itemize}

\section{Related Work}
\label{sec:relatedwork}


\subsection{Analytic BRDF Models}
\label{hyperbrdf-RW}
Analytic models are the most common representation for BRDFs. Classic BRDF models include Phong~\cite{blinn77}, Cook-Torrance~\cite{cooktorrance1982}, Ward~\cite{ward1992} and GGX~\cite{walter2007microfacet}. Following models have increased their reconstruction capabilities by combining analytic formulations with data-driven representations of some or all of their components~\cite{dupuy2015, ashikhmin2007, bagher2016}. Notably, the ABC model~\cite{low2012} has been shown to provide an accurate reconstruction of measured materials, while only requiring the fitting of a handful of tunable parameters. These models are usually fast at evaluation, easily editable, and present a low memory footprint. However, they usually rely on oversimplifications of the reflectance distribution shapes, and thus have a limited capacity for the reconstruction of complex real-world materials~\cite{ngan2005, guarnera2016}. Therefore, recent works have started exploring neural representations to overcome these limitations.



\subsection{Regression-based BRDF Estimation}
For a simpler representation of measured BRDFs, BRDF decomposition methods, such as PCA decomposition 
\cite{matusik2003data, nielsen2015optimal, serrano2018intuitive}, non-negative matrix factorization \cite{lawrence2004efficient, lawrence2006inverse}, Gaussian mixture \cite{sun2007interactive}, tensor decomposition \cite{bilgili2011general, tongbuasirilai2020compact} 
and non-parametric models~\cite{bagher2016non} have been proposed. The main limitation of PCA and factorization methods is  that they have a limited capacity to represent complex functions without overfitting to the training dataset. In contrast, our method can represent the complex BRDFs even with sparse samples while maintaining generalizability.


\paragraph{Deep learning for BRDF modeling.}

Multiple methods have been recently proposed for neural BRDF representation~\cite{rainer2019neural, hu2020deepbrdf, sztrajman2021neural, zheng2021compact, maximov2019deep, chen2021invertible, fan2021neural, cnf2023}. These methods usually offer a flexible representation, and thus are well fitted to encode the complex reflectance distributions of real-world measured materials. However, an accurate fitting of these methods usually requires lengthy optimizations and a very large number of sample measurements, typically from $8 \times 10^5$ to $1.5 \times 10^6$. \cite{maximov2019deep} learned materials with baked illumination via small fully-connected networks. NBRDF \cite{sztrajman2021neural} and CNF~\cite{cnf2023} leveraged neural fields to learn individual BRDF functions. Closer to our work, DeepBRDF~\cite{hu2020deepbrdf} and Neural Processes~\cite{zheng2021compact} introduce neural network architectures to learn a compressed latent space from a dataset of multiple materials. However, these methods only address the problem of compressing BRDF samples into a low dimensional space, hence overfitting to the dataset. Our method, on the other hand, also offers a generalizable approach for the reliable reconstruction of unseen materials from sparse and unstructured real-world reflectance measurements.


\subsection{Efficient BRDF Acquisition}
Realistic reflectance acquisition commonly requires a large amount of physical acquisition samples collected from different directions, making the process time-consuming and data intensive. To take fewer samples, hence reducing the BRDF capture time, optimization of a sample pattern with a linear statistical analysis of a database of BRDFs \cite{nielsen2015optimal} and the joint optimization of the sample pattern and a non-linear BRDF model~\cite{liu2023learning} have been proposed.

\paragraph{Spatially-varying BRDFs (SVBRDF):} For efficient SVBRDF capture, several methods based on multiplexing-based, also known as illumination-based, acquisition systems \cite{kang2018efficient, kang2019learning, ma2021free, ma2023opensvbrdf, tunwattanapong2013acquiring} have been proposed. A common approach has been to optimize the lighting patterns for efficient acquisition, followed by a BRDF fitting to an analytic model. Recent works have also leveraged deep learning architectures to learn a mapping from images to texture maps of analytic SVBRDF parameters~\cite{guo2021highlight, hui2017reflectance, deschaintre2018single, deschaintre2019flexible, martin2022materia, zhou2021adversarial,gao2019deep}. 

In contrast to those works, our focus is the reconstruction of spatially uniform BRDF that can accurately represent arbitrary complex materials. The works on spatially-varying BRDFs and efficient capture could be considered orthogonal to ours, and those methods could be potentially combined with ours. 


\subsection{Hypernetworks and GNFs}
The capacity of hypernetworks to dynamically output neural network weights, which allows models to adjust to input conditions, has drawn attention. Its promise in a variety of computer vision tasks, including dynamic network adaptation and generating neural implicit fields, has been demonstrated by recent works, such as HyperGAN \cite{ratzlaff2019hypergan} and Hyperdiffusion \cite{erkocc2023hyperdiffusion}.
The concept of a generalizable neural field has also been extensively applied to the reconstruction of neural radiance fields~\cite{wang2022attention, yang2023contranerf}, but not sufficiently studied in other domains.
These research efforts serve as our source of inspiration as we apply hypernetworks and GNFs to BRDF estimation, improving the model's adaptability to various material appearances.


\section{Methodology}

\begin{figure*}[t]
  \centering

   \includegraphics[width=\linewidth]{Chapters/hyperbrdf-figs/MainFig_13_11.pdf}
   \caption{During training, the set encoder and hypernetwork decoder are trained on a set of materials to predict the weights of hyponet (MLP) so that it can reconstruct the training set. The BRDF data is provided as a set of BRDF coordinates, $H_n,D_n$, and the corresponding reflectance values $f_r(H_n,D_n)$. To reconstruct a new material from a small set of BRDF reflectance samples, the trained set encoder and hypernetwork decoder are used to predict the weights of hyponet for the unknown material. Once those weights are known, we can query BRDF at any coordinates and for any new materials, conditioned on the embedding of their sampled BRDF values.}
   \label{fig:mainfig} \end{figure*}

Based on the observation that recent work misses a generalised and adaptable \gls{BRDF} representation, I introduce a novel representation for measured \gls{BRDF}s that learns compact embeddings of the \gls{BRDF}s with a hypernetwork model.

\subsection{Pre-processing}\label{sec:pre-proc}

Measured \gls{BRDF}s usually contain high dynamic range (\gls{HDR}) data, including arbitrarily high values, especially for the specular components. Hence, I pre-process the \gls{BRDF} data by applying a Log Relative Mapping~\cite{nielsen2015optimal} of the form:
\begin{equation}
  \rho' = \ln{\left(\frac{\rho + \epsilon}{\rho_{ref} + \epsilon} +1\right)}
  \label{eq:preprocess}
\end{equation}
where $\rho$ refers to the \gls{BRDF} values, $\epsilon = 0.002$ is a small constant value to avoid zero-division, and $\rho_{ref}$ is a reference \gls{BRDF} value for relative mapping. The reference \gls{BRDF} value is chosen to be the median value for each angle over the entire dataset.


We also observe that input parameterisation has a strong impact on the reconstruction quality since it guides the neural network model to learn different aspects of the reflectance function. Therefore, similar to NBRDF~\cite{sztrajman2021neural}, I express the \gls{BRDF}s as functions of the Cartesian vectors $H$ and $D$ in the Rusinkiewicz parameterisation \cite{rusinkiewicz1998new},
\textit{i.e.}, $\rho=f_r(H, D)$, where $H, D \in S^2$ indirectly encode the information about the incident and outgoing light directions.

Importantly, in this parameterization the directions of specular reflection have a single fixed representation as $H=(0,0,1)$, which provides an easier pattern for the network to learn than the traditional $\omega_i, \omega_o$ encoding.


\subsection{Hypernetwork}
\label{sec:hypernet}

Figure~\ref{fig:mainfig} shows the diagram of the hypernetwork model~\cite{sitzmann2020siren} for \gls{BRDF} representation, with three main components: 1) a set encoder that generates compact latent representations $Z$ of \gls{BRDF}s based on an arbitrary number of directional samples, 2) a hypernetwork decoder that decodes the latent to estimate the parameters of a neural field,
and 3) a neural field controlled by the decoder that represents the \gls{BRDF}s of the material, which we refer to as a hyponet, following the convention of prior work~\cite{sitzmann2020metasdf}.


\subsubsection{Set encoder} %add number of layers

The set encoder takes as input an arbitrary set of samples, $n=1..N$, taken from a \gls{BRDF} measurement. Each measurement consists of directional coordinates ${H_n, D_n}$, given in the Rusinkiewicz parameterisation~\cite{rusinkiewicz1998new}, and their corresponding \gls{BRDF} values $f_r(H_n,D_n)$. The set encoder is composed of four fully-connected layers with two hidden layers of feature size 128 for each. The input is the concatenation of \gls{BRDF} values $(N, 3)$ and coordinates $(N, 6)$. The activation function applied after each layer is \gls{ReLU}. Each sample is encoded into a 40-dimensional embedding, and the sample set is reduced to a single embedding by applying a symmetric operation (averaging).
% \FZ{Permutation invariant operations are common in point set networks. Consider referring to the literature to motivate our design over alternatives.}
The use of a set encoder, which is commonly adopted in point set networks \cite{zaheer2017deepsets}, ensures permutation invariance and provides a high degree of flexibility in terms of the input, enabling the encoding of \gls{BRDF}s with an arbitrary number of data-points, irregularly sampled, and in no pre-defined order.


\subsubsection{Hypernetwork decoder and hyponet} %add number of layers and type
The hypernetwork decoder converts the embeddings from the set encoder into the weights of the hyponet that represents the \gls{BRDF}s of a single material. The hypernetwork decoder is composed of 10 blocks of a fully-connected neural network with three layers. Each block outputs the corresponding weights and biases of hyponet. The hyponet consists of five fully-connected layers with input layer of size six for coordinates, three hidden layers of size 60 for each, output layer of size three for \gls{BRDF} values. The neural representation of materials, hyponet, follows the structure of NBRDF~\cite{sztrajman2021neural}, but replaces the exponential activation (\gls{ELU}) in the last layer with a \gls{ReLU} activation due to \gls{BRDF} properties ($\rho \ge 0$). This network provides a continuous representation of a \gls{BRDF}, and has been shown to provide state-of-the-art reconstructions of measured \gls{BRDF}s, with performances competitive with the fastest analytic \gls{BRDF} models.


\subsection{Training}
\label{sec:traindet}


I train the hypernetwork by optimizing the following loss, which consists of a reconstruction term $\mathcal{L}_\text{rec}$ and two regularization terms $\mathcal{L}_\text{weights}$ and $\mathcal{L}_\text{latent}$ for the hyponet weights $w$ and the latent embeddings $z$ ~\cite{ha2017hypernetworks}:
\begin{equation}
    \mathcal{L} = \mathcal{L}_\text{rec} +
              \lambda_1 \underbrace{\frac{1}{W} \sum_{j=1}^W w^2_j}_{\mathcal{L}_\text{weights}} +
              \lambda_2 \underbrace{\frac{1}{Z} \sum_{k=1}^Z z^2_k}_{\mathcal{L}_\text{latent}}
    \label{eq:loss}
\end{equation}

The reconstruction loss is defined as the mean squared error between the cosine weighted predicted and ground-truth \gls{BRDF} values:

\begin{equation}
    \mathcal{L}_{\text{rec}} = \sum_{n=1}^{N}\sum_{m=1}^{M}\frac{\left|\left|\rho^{\text{pred}}_{n, m} \cos{\theta_{n, m}} - \rho^{\text{true}}_{n, m} \cos{\theta_{n, m}}\right|\right|_{2}}{NM}
    \label{eq:Lrec}
\end{equation}

where $\rho^{\text{pred}}_{n, m}$ and $\rho^{\text{true}}_{n, m}$ indicate the predicted and ground truth \gls{BRDF} values of the $n$-th sample of the $m$-th material, both processed as described in Eqn \ref{eq:preprocess}, and $\theta$ measures the angle between the incident ray and the surface normal. The cosine term weighs \gls{BRDF}s based on the assumption of uniform incoming radiance and leads to more visually accurate results \cite{ngan2005experimental}.


I train the network for 80 epochs with 1,458,000 samples per material. It takes around 15 minutes with an NVIDIA A100 80GB GPU support. 


\paragraph{Inference:}
When inferring the reflectance values, HyperBRDF first estimates hyponet weights from sparse samples of a test material, which takes around 0.01 seconds without GPU. Later, I feed query coordinates to the hyponet to predict the \gls{BRDF} values of the material. With the conversion of the predicted \gls{BRDF} into a renderable format, this process takes around 9 seconds without GPU. The continuous representation of \gls{BRDF}s with the hyponet offers a nonlinear built-in interpolation and, hence, accurately reconstruct unseen materials from even a few samples.
\section{Experiments}\label{sec:exp}


\subsection{Datasets and baselines}

To show the effectiveness of the proposed method, I use MERL \cite{Matusik2003jul} and RGL (51 isotropic materials) \cite{dupuy2018adaptive} datasets, which are the most commonly used \gls{BRDF} datasets that include isotropic materials. The MERL dataset \cite{Matusik2003jul} consists of 100 real-world materials, covering a wide range of appearances. Each material includes reflectance measurements from a dense set of directions, parameterized as the spherical angles ($\theta$, $\phi$) of the $H$ and $D$ vectors from the Rusinkiewicz parameterization \cite{rusinkiewicz1998new}. Each colour channel has a resolution of $90 \times 90 \times 180$, leading to 1,458,000 reflectance measurements. 



\subsection{Sparse BRDF reconstruction}\label{sec:brdf_rec}

To understand the reconstruction capacity of HyperBRDF, I first train the model with all available samples (1,458,000). I observe that reducing the number of samples by around half (640,000) results in a similar performance (Table \ref{table: ours_large_samples}). For comparison with baselines, I train the hypernetwork model with 80 MERL materials that are randomly selected. I leave the remaining 20 materials for testing. I also train HyperBRDF with a mixed dataset of 80 MERL materials and 51 isotropic RGL materials. I later test the trained model on the same test dataset (20 MERL materials). I apply a separate log relative mapping to RGL materials by computing the median over the isotropic RGL dataset.
 
I qualitatively compare the results against the ground truths through renderings of the materials. The renderings are obtained by a Mitsuba renderer with an environment map illumination. HyperBRDF can capture the diffuse colours of varying unseen appearances even for the materials with specular components. 


The main advantage of the hypernetwork architecture is that it is flexible in the number of samples fed to the network. That is, we can reconstruct unseen materials with fewer samples than the sample number used during training. Thanks to its built-in nonlinear interpolation that comes from the hyponet, I obtain high quality reconstruction results with fewer samples. 

Considering the hypernetwork architecture, we understand that the gap between reconstruction results with sparse samples relies on the embeddings $z$ (latent vectors). I hence analyse the embeddings for the materials reconstructed with varying number of samples. Figure \ref{fig:tsne-vis-imputation} illustrates the t-SNE clustering of the test embeddings with different number of samples. It is visible that the embeddings of the same material reconstructed with different sample sizes lie close in the t-SNE space.

\begin{figure}[t]
  \centering
  % \fbox{\rule{0pt}{2in} \rule{0.9\linewidth}{0pt}}
   \includegraphics[width=0.8\linewidth]{Chapters/hyperbrdf-figs/tsne6_2_den1-cropped-compressed.pdf}
   \caption{t-SNE clustering of the test embeddings with different sample sizes, including $N=8, 40, 160, 4\,000, 40\,000, 640\,000$.}

   \label{fig:tsne-vis-imputation}
\end{figure}


\subsubsection{Qualitative comparison}\label{sec:qual_comp}
\begin{figure*}[t]
  \centering
%\adjustbox{trim={0.\width} {.\height} {0.89\width} {.\height},clip}%
  %{\includegraphics[width=0.9\linewidth]{Chapters/hyperbrdf-figs/imp_comp_upt_3vals.pdf}}
%\adjustbox{trim={0.113\width} {.\height} {0.596\width} {.\height},clip}%
  %{\includegraphics[width=0.9\linewidth]{Chapters/hyperbrdf-figs/imp_comp_upt_3vals.pdf}}
%\adjustbox{trim={0.41\width} {.\height} {0.298\width} {.\height},clip}%
  %{\includegraphics[width=0.9\linewidth]{Chapters/hyperbrdf-figs/imp_comp_upt_3vals.pdf}}
%\adjustbox{trim={0.707\width} {.\height} {0.\width} {.\height},clip}%
 % {\includegraphics[width=0.9\linewidth]{Chapters/hyperbrdf-figs/imp_comp_upt_3vals.pdf}}

%\adjustbox{trim={0.\width} {.\height} {0.89\width} {.06\height},clip}%
  %{\includegraphics[width=0.9\linewidth]{Chapters/hyperbrdf-figs/imp_comp_upt_3vals_bad.pdf}}
%\adjustbox{trim={0.113\width} {.\height} {0.596\width} {.06\height},clip}%
  %{\includegraphics[width=0.9\linewidth]{Chapters/hyperbrdf-figs/imp_comp_upt_3vals_bad.pdf}}
%\adjustbox{trim={0.41\width} {.\height} {0.298\width} {.06\height},clip}%
  %{\includegraphics[width=0.9\linewidth]{Chapters/hyperbrdf-figs/imp_comp_upt_3vals_bad.pdf}}
%\adjustbox{trim={0.707\width} {.\height} {0.\width} {.06\height},clip}%
 % {\includegraphics[width=0.9\linewidth]{Chapters/hyperbrdf-figs/imp_comp_upt_3vals_bad.pdf}}
  
  \adjustbox{trim={0.07\width} {.\height} {0.816\width} {.\height},clip}%
  {\includegraphics[width=0.9\linewidth]{Chapters/hyperbrdf-figs/qual_comp_ggx_2.pdf}}
\adjustbox{trim={0.184\width} {.\height} {0.41\width} {.\height},clip}%
{\includegraphics[width=0.9\linewidth]{Chapters/hyperbrdf-figs/qual_comp_ggx_2.pdf}}
\adjustbox{trim={0.593\width} {.\height} {0.\width} {.\height},clip}%
  {\includegraphics[width=0.9\linewidth]{Chapters/hyperbrdf-figs/qual_comp_ggx_2.pdf}}
 %{\includegraphics[width=0.9\linewidth]{Chapters/hyperbrdf-figs/qual_comp_ggx_2.pdf}}
\\
\hspace{1.9cm}{\includegraphics[width=0.73\linewidth]{Chapters/hyperbrdf-figs/N_labels2.pdf}}
   \caption{Qualitative comparison results for reconstruction with small sample sizes. Thanks to the prior that the hypernetwork model learns for material appearance through training, it can accurately estimate theBRDFs of unseen materials and preserve the colours better than the baselines.}  

   \label{fig:imp_comp_upt}
\end{figure*}

\begin{figure*}[t]
  \centering
    {\includegraphics[width=0.35\linewidth]{Chapters/hyperbrdf-figs/legend.png}}\\
  {\includegraphics[width=0.32\linewidth, height=3.4cm]{Chapters/hyperbrdf-figs/PSNR_ggx.pdf}}
  {\includegraphics[width=0.32\linewidth, height=3.4cm]{Chapters/hyperbrdf-figs/DeltaE_ggx.pdf}}
  {\includegraphics[width=0.32\linewidth, height=3.4cm]{Chapters/hyperbrdf-figs/MAE_ggx.pdf}}
  {\includegraphics[width=0.32\linewidth, height=3.4cm]{Chapters/hyperbrdf-figs/RMSE_ggx.pdf}}
    {\includegraphics[width=0.32\linewidth, height=3.4cm]{Chapters/hyperbrdf-figs/RAE_ggx.pdf}}
    {\includegraphics[width=0.32\linewidth, height=3.4cm]{Chapters/hyperbrdf-figs/SSIM_ggx.pdf}}
   \caption{Average \gls{PSNR}, Delta E (CIE 2000), \gls{SSIM}, MAE, RMSE, and RAE results across different sample sizes. }
   \label{fig:imp_plots}
\end{figure*}



I first compare the results with the ground truth renderings for varying sample sizes. Figure \ref{fig:imp_comp_upt} shows that the hypernetwork can still reconstruct test materials even with $N = 40$ samples. However, as I reduce the sample number to 10, the reconstruction quality highly decreases. 

During inference time, hypernetwork fits the samples to a \gls{BRDF}. Even with a few samples, it can still accurately reconstruct test materials. Its success comes from the fact that it learns a prior for material appearance through training with a dataset of multiple materials. Therefore, I validate this generalizable approach by comparing the hypernetwork with an analytic model, GGX  \cite{walter2007microfacet}, NBRDF \cite{sztrajman2021neural} and PCA with Log Relative Mapping (IPCA) \cite{nielsen2015optimal}.

Sparse GGX results are obtained by fitting to the sparse number of measured \gls{BRDF} samples using a non-linear optimisation (L-BFGS-B method) with Log $\mathcal{L}_{1}$ loss (same loss used in NBRDF). For the results with all samples, I used the dj\_brdf mitsuba plug-in \cite{dupuy2015photorealistic}. NBRDF \cite{sztrajman2021neural} is designed to implicitly represent an individual \gls{BRDF}. It can reconstruct materials with very high accuracy (\gls{SSIM} $\approx$ 0.995) when the sample size is high. It first trains the network on a material and then estimates the \gls{BRDF} values of the same material. For a fair comparison, I first fit an NBRDF model to a material \gls{BRDF} with the query sample size, then estimate the function with the same sample size. 

I also compare the results with a PCA-based strategy (IPCA). To represent the \gls{BRDF} data, PCA-based methods  \cite{matusik2003data, ngan2006image} construct a matrix ${A} \in \mathbb{R}^{m \times n}$, where n = $180 \times 90 \times 90 \times 3$ = 4,374,000 is the feature number, m = 80 is sample number. The matrix is later decomposed into its principal components via Singular Value Decomposition. PCA itself struggles with the decomposition of high dynamic range data. Therefore, the Log Relative Mapping (IPCA) is proposed~\cite{nielsen2015optimal}, which is defined in the pre-processing step (Section \ref{sec:pre-proc}). This improves the reconstruction quality, offering competitive reconstructions against HyperBRDF as shown in Figure \ref{fig:imp_comp_upt}.

For all methods including HyperBRDF, samples are drawn from a uniform distribution over Rusinkiewicz coordinates. For IPCA results, I split train and test dataset in the same way as HyperBRDF, i.e., 80 MERL materials for train and 20 for test, and use all available samples for learning the principal components. To reconstruct the test materials from sparse samples, I first run a least-squared-error optimisation and predict the weights of the material for the corresponding principal components. To decide the number of components, I ran an additional ablation study, where 
I computed the average mean squared error over the test dataset for the reconstruction from sparse samples ($N = 40$). We observed that ($N_{PC} = 8$) gives the minimum error on the test dataset. Hence, I choose the number of principal components as $N_{PC} = 8$ and keep it same for sparse sampling results.

 
I keep the range for sparse sample numbers between $N = 40$ and $N = 4000$. Figure \ref{fig:imp_comp_upt} shows the rendering results for the minimum and maximum number of this range. Compared to GGX, IPCA and NBRDF, the hypernetwork can capture the appearances more precisely. Although this approach has difficulty estimating specular components (last two rows in Figure~\ref{fig:imp_comp_upt}), overall it offers reconstructions with much higher accuracy. 

I observe that the diffuse colours of some reconstructed materials by NBRDF can be completely off (natural-209, teflon) due to function fitting with sparse samples. Hypernetwork, on the other hand, can preserve diffuse colours better thanks to the prior it learns through training.


\subsubsection{Quantitative evaluation}

I compare the proposed method quantitatively with the aforementioned techniques in multiple image-based error metrics through rendering results. The metrics include \gls{PSNR}, Delta E (CIE 2000), \gls{SSIM}, \gls{MAE}, root mean square error (RMSE), and relative absolute error (RAE). I take the average over the test dataset for each metric. I plot the metric results across five different sample sizes (40, 160, 400, 2000, 4000). Since I optimize IPCA on the test dataset for the $N = 40$ case, it offers more competitive results. Nevertheless, HyperBRDF can reconstruct the \gls{BRDF}s of unseen materials more precisely. Figure \ref{fig:imp_plots} shows that across all sample sizes, the hypernetwork attains superior performance in terms of \gls{PSNR}, Delta E (CIE 2000) and \gls{SSIM}. 

Additionally, Table \ref{table: ours_diff_samples} shows that expanding the training set, even with materials captured from a different point-based setup, helps improve the performance of the hypernetwork. Also, note that compared to MERL materials, the colours of RGL materials are more saturated, which could explain a slight increase in Delta E.


 \begin{table*}[ht]
    \centering
    \caption{Hypernetwork sparse reconstruction - Average metric results across varying sample sizes ($N$) over the test set. We highlight \colorbox{blue!25}{best} and \colorbox{orange!25}{second best} results.}
    
    {\begin{tabular}{l@{\hskip 0.4in}c@{\hskip 0.2in}c@{\hskip 0.2in}c@{\hskip 0.1in}|@{\hskip 0.1in}c@{\hskip 0.2in}c@{\hskip 0.2in}c}\toprule
    

& \multicolumn{3}{c}{MERL} & \multicolumn{3}{c}{MERL + RGL}
\\\cmidrule(lr){2-4}\cmidrule(lr){5-7}
% \toprule
  $N$ & \gls{PSNR}\textuparrow & Delta E\textdownarrow & SSIM\textuparrow & PSNR\textuparrow & Delta E\textdownarrow & \gls{SSIM}\textuparrow \\

 \toprule

$40$ & 29.581 & 3.189 & 0.968 & 30.018 & 3.112 & 0.963\\
$160$ & 31.341 & 2.681 & 0.973 & 31.929 & 2.454 & 0.962\\
$400$ & 32.743 & 2.272 & \cellcolor{blue!25} 0.979 & 33.855 & 2.432 & 0.977\\
$2000$ & \cellcolor{blue!25} 34.527 & \cellcolor{orange!25}2.256 & \cellcolor{blue!25} 0.979 & \cellcolor{blue!25} 34.527 & \cellcolor{orange!25} 2.243 & \cellcolor{blue!25} 0.983\\
$4000$ & \cellcolor{orange!25} 33.170 &  \cellcolor{blue!25} 2.138 & \cellcolor{orange!25} 0.977 & \cellcolor{orange!25} 34.355 & \cellcolor{blue!25} 2.166 & \cellcolor{orange!25} 0.982\\

\bottomrule
    \end{tabular}\par}
    \label{table: ours_diff_samples}
\end{table*}

\paragraph{Sparse and full reconstruction results:}

\begin{table*}[ht]
    \centering
    \caption{Hypernetwork - Average metric results across varying sample sizes ($N$) over the test set (Sparse and full reconstruction of unseen materials).}

    \resizebox{0.75\linewidth}{!}{%
    {\begin{tabular}{l@{\hskip 0.2in}c@{\hskip 0.2in}c@{\hskip 0.2in}c@{\hskip 0.2in}c@{\hskip 0.2in}c@{\hskip 0.2in}c}\toprule

 Metrics/$N$ & $8$ & $40$ & $4000$ & $40\,000$ & $640\,000$ & $1\,458\,000$\\
 \toprule
PSNR\textuparrow & 23.090 & 29.822 & 33.170 & 33.019 & \cellcolor{blue!25}33.166 & \cellcolor{orange!25} 33.128 \\
Delta E\textdownarrow & 5.948 & 3.086 & 2.138 & \cellcolor{orange!25} 2.118 & \cellcolor{blue!25}2.117 & 2.181 \\
SSIM\textuparrow & 0.927 & 0.969 & 0.977 & \cellcolor{blue!25} 0.979 & \cellcolor{blue!25} 0.979 & \cellcolor{orange!25}0.978 \\
MAE\textdownarrow & 13.167 & 5.306 & 3.642 & \cellcolor{orange!25} 3.568 & \cellcolor{blue!25}3.492 & 3.574 \\
RMSE\textdownarrow & 22.293 & 9.601 & 6.791 & \cellcolor{orange!25} 6.618 & \cellcolor{blue!25}6.507 & 6.740 \\
RAE\textdownarrow & 0.339 & 0.134 & 0.093 & \cellcolor{orange!25} 0.091 & \cellcolor{blue!25}0.089 & \cellcolor{orange!25} 0.091 \\
\bottomrule
    \end{tabular}\par}}
    \label{table: ours_large_samples}
\end{table*}


\subsubsection{Comparison with casual capture setups}
Casual \gls{BRDF} estimation methods reconstruct \gls{BRDF}s from a captured material image; hence, they struggle with disentangling the material from unknown illumination, often requiring a data-driven prior. As a result, most recent papers make simplifying assumptions about the material, fitting to a Phong or an analytic model. Even though iBRDF \cite{chen2021invertible} explores fitting to real-world reflectance measurements, HyperBRDF still overwhelmingly outperforms their reconstruction quality with only 160 samples and above. When iBRDF is run with the same rendering setup on the same test dataset, we observe that even HyperBRDF's $N=160$ case overall offers more accurate results. Thus, casual capture remains highly impractical for professional applications mentioned earlier. Also, note that iBRDF trains their model with ground-truth measured \gls{BRDF} values, and hence their requirements for supervision are no less than HyperBRDF. 


%In casual \gls{BRDF} estimation, the main bottleneck lies in decoupling unknown illumination from the material, often necessitating a data-driven prior, thus is not our main focus. Consequently, most recent papers resort to assuming a Phong or analytic model for simple materials. Although iBRDF \cite{chen2021invertible} relaxed this constraint, we significantly outperform their reconstruction quality with only 160 samples and above. We ran iBRDF with the same rendering setup on our test dataset and observe that even our $N=160$ case overall offers more accurate results. Thus, casual capture remains highly impractical for professional applications mentioned earlier. We also highlight that iBRDF relies on ground-truth measured \gls{BRDF} values for training, and hence requires no less supervision than ours. 

\begin{table}[ht]
    \centering
    \caption{Comparison with iBRDF \cite{chen2021invertible}  - Average metric results over the renderings of the entire MERL dataset. We highlight \colorbox{blue!25}{best} and \colorbox{orange!25}{second best} results.}

    {%
    {\begin{tabular}{l@{\hskip 0.5in}c@{\hskip 0.3in}c@{\hskip 0.3in}c}\toprule


  &  PSNR \textuparrow & Delta E \textdownarrow & SSIM \textuparrow \\
 \toprule
 HyperBRDF ($N = 400$) & \cellcolor{blue!25} 32.74  & \cellcolor{blue!25}  0.979  & \cellcolor{blue!25} 2.272\\
 iBRDF (Best) \cite{chen2021invertible} & \cellcolor{orange!25}  31.46 &  0.954 & 2.879\\
 HyperBRDF ($N = 160$) & 31.34  & \cellcolor{orange!25}  0.973  & \cellcolor{orange!25} 2.681\\

\bottomrule
    \end{tabular}\par}}
    \label{table: oursvsibrdf}
\end{table}


\subsection{Compression}\label{sec:compression}
The high capacity of the hypernetwork also allows the compression of the densely sampled \gls{BRDF} data into low-dimensional latent representations. The hypernetwork can process highly compact \gls{BRDF} embeddings, and once decoded, reconstructs the \gls{BRDF} data precisely. I compare HyperBRDF with Neural Processes (NPs) \cite{zheng2021compact}, the state-of-the-art \gls{BRDF} compression method, and show that the hypernetwork model overall performs better in all three metrics. 


To compare this method with NPs, I overfit HyperBRDF to the mixed dataset of MERL and isotropic RGL materials, consisting of 151 materials in total. Since the latent dimension of NPs is 7D, I also train the hypernetwork with 7D latent space. It is worth mentioning that NPs cause invalid sample values in certain MERL materials, blacking some parts of the renderings. In contrast, this method consistently decompresses materials with high reconstruction accuracy (Table \ref{table: oursvsnps}). Additional rendering results can be found in Appendix \ref{hyperbrdf:add_res}.

\begin{table}[ht]
    \centering
    \caption{Compression - Average metric results over the renderings of the entire MERL dataset. We highlight \colorbox{blue!25}{best} and \colorbox{orange!25}{second best} results.}

    {%
    {\begin{tabular}{l@{\hskip 0.5in}c@{\hskip 0.3in}c@{\hskip 0.3in}c}\toprule


  &  \gls{PSNR} \textuparrow & Delta E \textdownarrow & \gls{SSIM} \textuparrow \\
 \toprule
 HyperBRDF (40D) & \cellcolor{blue!25} 47.682 & \cellcolor{blue!25} 0.567 & \cellcolor{blue!25} 0.994\\
 HyperBRDF (7D) & \cellcolor{orange!25} 47.492 & \cellcolor{orange!25} 0.574 & \cellcolor{blue!25} 0.994\\
 NPs & 46.125 & 2.424 & 0.935\\
 IPCA & 29.892 & 3.315 & 0.979\\

\bottomrule
    \end{tabular}\par}}
    \label{table: oursvsnps}
\end{table}


\begin{figure*}[ht]
  \centering
  {\includegraphics[width=\linewidth]{Chapters/hyperbrdf-figs/compression_comp1.pdf}}
   \caption{Reconstruction results for BRDF compression (GT: ground truths).}
   \label{fig:comp-fig}
\end{figure*}


\subsection{BRDF editing}\label{sec:brdf-editing}

Compared to analytic \gls{BRDF}s that have a fixed number of parameters to tweak, editing measured \gls{BRDF}s is rather a nontrivial task due to irregular data structure. On the other hand, the hypernetwork is capable of \gls{BRDF} editing thanks to its representation of materials in a low-dimensional space. We can easily reconstruct various appearances by linearly interpolating between the embeddings of two different materials. Figure \ref{fig:interpolation} shows newly-reconstructed materials through linear interpolation between two different embeddings from the reconstructed MERL materials. 


\begin{figure*}[ht]
  \centering
  % \fbox{\rule{0pt}{2in} \rule{0.9\linewidth}{0pt}}
   \includegraphics[width=\linewidth]{Chapters/hyperbrdf-figs/interpolation_extended.pdf}

   \caption{BRDF editing through linear interpolation between the embeddings of two materials.}
   \label{fig:interpolation}
\end{figure*}

\subsection{Ablation studies}\label{sec:abl}
\subsubsection{Latent dimension}
I experimented with the dimension size of the latent representations $Z$ on the test dataset for the reconstruction with all available samples and observed that 40-dimensional embeddings give the minimum average mean squared error (the first row of Table \ref{table: z_abl}). Therefore, I chose the latent space size to be 40. However, I also observed that these can change with the number of sparse samples. For consistency, I kept it same for all the reconstruction results. 

\subsubsection{Log relative mapping (LRM)}\label{sec:lrm}
I apply Log Relative Mapping (LRM) \cite{nielsen2015optimal} to the \gls{BRDF} data before feeding it the hypernetwork, as described in Section \ref{sec:pre-proc}. I illustrate that LRM improves the reconstruction quality for both PCA and hypernetwork results in Table \ref{table: comparison results} and Figure \ref{fig:qual_comp}. In the results, IPCA refers to the mapping-applied version of PCA, and "HyperBRDF (No LRM)" to HyperBRDF without LRM.


\subsubsection{Cosine weighting}
Similar to \citeauthor{ngan2005experimental} \cite{ngan2005experimental}, I also weigh the \gls{BRDF}s with a cosine term based on the assumption of uniform incoming radiance and observed that it offers higher accuracy (Table \ref{table: cos_abl}).

\begin{table}[ht]
    \centering
    \caption{Average metric results over the renderings of the test set (20 MERL materials).}

    {\begin{tabular}{l@{\hskip 0.3in}c@{\hskip 0.3in}c@{\hskip 0.3in}c}\toprule
    % \resizebox{0.5\linewidth}{!}{%
    % {\begin{tabular}{ccc}\toprule

 &  No Cosine &  Cosine\\
 \toprule
 PSNR\textuparrow & 33.077 & \cellcolor{blue!25} 33.166 \\
Delta E\textdownarrow & 2.233 & \cellcolor{blue!25} 2.117 \\
SSIM\textuparrow & 0.974 & \cellcolor{blue!25} 0.979 \\
MAE\textdownarrow & 3.908 & \cellcolor{blue!25} 3.492 \\
RMSE\textdownarrow & 7.370 & \cellcolor{blue!25} 6.507 \\
RAE\textdownarrow & 0.098 & \cellcolor{blue!25} 0.089 \\
\bottomrule
    \end{tabular}\par}
    \label{table: cos_abl}
\end{table}

\subsubsection{Principal component analysis (PCA)}
In Section \ref{sec:qual_comp}, I briefly discussed the ablation study I ran to choose the number of principal components for the best-performing IPCA results. Here, I elaborate on the effect of the number of principal components on the performance of IPCA. 


\paragraph{Number of Principal Components:}
I analyzed the optimal number of principal components by initially choosing the same numbers as we did for the ablation study of the latent dimension. Looking at Table \ref{table: z_abl}, I observe that $N_{PC}$ gives the minimum error for sparse cases, where the number of samples $N$ is 40. I further validated the choice $N_{PC} = 8$ by running IPCA with the number of principal components ranging from 1 to 16 with an incremental change of 1. Figure \ref{fig:ipca_opt} illustrates that $N_{PC} = 8$ still gives the minimum mean squared error (MSE) averaged over the test set.


\begin{figure}[ht]
  \centering

  {\includegraphics[width=0.6\linewidth]{Chapters/hyperbrdf-figs/ipca_opt_q_40_cropped.pdf}}
   \caption{IPCA optimization for the number of principal components.}
   \label{fig:ipca_opt}
\end{figure}

\begin{table*}[ht]
    \centering
    \caption{Average mean squared errors for varying latent space dimensions (first row) and number of principal components (second row).}

    \resizebox{0.9\linewidth}{!}{%
    {\begin{tabular}{l@{\hskip 0.1in}c@{\hskip 0.1in}c@{\hskip 0.1in}c@{\hskip 0.1in}c@{\hskip 0.1in}c@{\hskip 0.1in}c@{\hskip 0.1in}c@{\hskip 0.1in}c}\toprule

 Methods/Dimension & 8 & 20 & 30 & 40 & 50 & 60 & 70 & 80\\
 \toprule
  HyperBRDF & 0.073 & \cellcolor{orange!25} 0.046 & 0.062 & \cellcolor{blue!25} 0.045 & 0.051 & 0.053 & 0.049 & \cellcolor{orange!25} 0.046\\
 IPCA & \cellcolor{blue!25} 0.131 & 0.278 & 12.288 & \cellcolor{orange!25} 0.507 & 0.878 & 0.883 & 0.965 & 1.124\\
 
\bottomrule
    \end{tabular}\par}}
    \label{table: z_abl}
\end{table*}


\subsection{Scene renderings}
Furthermore, I rendered various scenes \cite{resources16} using HyperBRDF's reconstructed materials, including sparse reconstruction, compression and interpolation. The details about the reconstructed materials are as follows (Figure \ref{fig:scene-render}): \textbf{\textit{teapot:}} steel. \textbf{\textit{dragons:}} interpolation of delrin and green-metallic-paint, white marble (dragon paint), silver-metallic-paint3 (ground), dark-red-paint (cloth). \textbf{\textit{cars:}} chrome steel, gold metallic paint3, specular-red-phenolic (car paint), exterior car parts (aluminium), inner wheel (alumina-oxide). \textbf{\textit{kitchen:}} cupboards (natural-209), utensils (chrome), handles, pot, microwave and cooker (two-layer-silver), extractor hood (aluminium), pot and kettle handles (black-obsidian), kettle paint (dark-red-paint), tea towel and cushions (yellow-paint), radio and lamp (polyethylene). \textbf{\textit{living room:}} sofa, coffee table, side tables, wall book shelf (pure-rubber), cushions (green-metallic-paint), twigs (natural-209), legs (dark-specular-fabric).
\textbf{\textit{sofas:}} violet rubber and green latex (sofa cover). 

\begin{figure}[ht]
  \centering
  {\includegraphics[width=\linewidth]{Chapters/hyperbrdf-figs/SceneRenderings1.pdf}}
   \caption{Scene renderings with materials reconstructed by HyperBRDF.}
   \label{fig:scene-render}
\end{figure}

\subsection{Limitations and future work}\label{sec:limits}
\paragraph{Specular components:} HyperBRDF struggles with the estimation of specular components as shown in Figure~\ref{fig:imp_comp_upt} (last two rows). It is likely because of the high gap between the values of diffuse that are close to zero and the values of specular components that are arbitrarily high. A separate estimation pipeline for each component within the network can help improve the results.


\paragraph{BRDF editing:} The \gls{BRDF} editing approach is rather non-intuitive with an interpolation approach. Finding a map between the embeddings and certain attributes of the materials, such as diffuse/specular colours or haziness, can lead to more interactive \gls{BRDF} editing. I plan to map the latent space to material parameter space so that users can easily edit materials through more interpretable parameters.

\paragraph{SVBRDF representations:} In this work, I focused on the task of generalizable neural representations for spatially uniform \gls{BRDF}s. As the future work, I plan to extend HyperBRDF to SVBRDF representations.

\section{Summary}\label{sec:conc}

Photorealistic rendering can be achieved through \gls{PBR}  systems that provide a more precise portrayal of light-material interactions. A \gls{PBR}  system requires an accurate representation of \gls{BRDF}s, which can be accomplished either through analytic models that quantify \gls{BRDF}s using a few parameters or through real-world measurements captured using professional setups. While real-world \gls{BRDF} measurements offer highly accurate material appearance representations, they demand extensive capture times and specialised equipment. This chapter introduces a hypernetwork model for neural \gls{BRDF} representation, designed for point-based \gls{BRDF} acquisition setups. By representing discrete \gls{BRDF} values as a continuous generalisable neural field, HyperBRDF achieves (1) accurate reconstructions of unseen materials from sparse sample sets and (2) compression of dense \gls{BRDF} values into compact latent representations. With a set encoder enabling arbitrary input samples and a neural field (hyponet) providing nonlinear interpolation, the hypernetwork model outperforms baselines in capturing \gls{BRDF}s of test materials. Additionally, I demonstrated the model's compression capabilities on the entire MERL dataset, yielding consistent results across various materials. Integrated with the \gls{PBR} systems and capture setups, HyperBRDF has the potential to pave the way for the immersive realism that virtual world creators highly anticipate.

\textbf{Acknowledgements.} I would like to thank all the co-authors of the associated paper for their invaluable help. In this chapter, I have included figures contributed by Alejandro Sztrajman (main figure, GGX, and iBRDF results) and Chenliang Zhou (t-SNE figure) to ensure the completeness of the narrative.

