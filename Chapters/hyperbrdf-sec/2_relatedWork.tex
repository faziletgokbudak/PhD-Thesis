\section{Related Work}
\label{sec:relatedwork}


\subsection{Analytic BRDF Models}
Analytic models are the most common representation for BRDFs. Classic BRDF models include Phong~\cite{blinn77}, Cook-Torrance~\cite{cooktorrance1982}, Ward~\cite{ward1992} and GGX~\cite{walter2007microfacet}. Following models have increased their reconstruction capabilities by combining analytic formulations with data-driven representations of some or all of their components~\cite{dupuy2015, ashikhmin2007, bagher2016}. Notably, the ABC model~\cite{low2012} has been shown to provide an accurate reconstruction of measured materials, while only requiring the fitting of a handful of tunable parameters. These models are usually fast at evaluation, easily editable, and present a low memory footprint. However, they usually rely on oversimplifications of the reflectance distribution shapes, and thus have a limited capacity for the reconstruction of complex real-world materials~\cite{ngan2005, guarnera2016}. Therefore, recent works have started exploring neural representations to overcome these limitations.



\subsection{Regression-based BRDF Estimation}
For a simpler representation of measured BRDFs, BRDF decomposition methods, such as PCA decomposition 
\cite{matusik2003data, nielsen2015optimal, serrano2018intuitive}, non-negative matrix factorization \cite{lawrence2004efficient, lawrence2006inverse}, Gaussian mixture \cite{sun2007interactive}, tensor decomposition \cite{bilgili2011general, tongbuasirilai2020compact} 
and non-parametric models~\cite{bagher2016non} have been proposed. The main limitation of PCA and factorization methods is  that they have a limited capacity to represent complex functions without overfitting to the training dataset. In contrast, our method can represent the complex BRDFs even with sparse samples while maintaining generalizability.


\paragraph{Deep learning for BRDF modeling.}

Multiple methods have been recently proposed for neural BRDF representation~\cite{rainer2019neural, hu2020deepbrdf, sztrajman2021neural, zheng2021compact, maximov2019deep, chen2021invertible, fan2021neural, cnf2023}. These methods usually offer a flexible representation, and thus are well fitted to encode the complex reflectance distributions of real-world measured materials. However, an accurate fitting of these methods usually requires lengthy optimizations and a very large number of sample measurements, typically from $8 \times 10^5$ to $1.5 \times 10^6$. \cite{maximov2019deep} learned materials with baked illumination via small fully-connected networks. NBRDF \cite{sztrajman2021neural} and CNF~\cite{cnf2023} leveraged neural fields to learn individual BRDF functions. Closer to our work, DeepBRDF~\cite{hu2020deepbrdf} and Neural Processes~\cite{zheng2021compact} introduce neural network architectures to learn a compressed latent space from a dataset of multiple materials. However, these methods only address the problem of compressing BRDF samples into a low dimensional space, hence overfitting to the dataset. Our method, on the other hand, also offers a generalizable approach for the reliable reconstruction of unseen materials from sparse and unstructured real-world reflectance measurements.


\subsection{Efficient BRDF Acquisition}
Realistic reflectance acquisition commonly requires a large amount of physical acquisition samples collected from different directions, making the process time-consuming and data intensive. To take fewer samples, hence reducing the BRDF capture time, optimization of a sample pattern with a linear statistical analysis of a database of BRDFs \cite{nielsen2015optimal} and the joint optimization of the sample pattern and a non-linear BRDF model~\cite{liu2023learning} have been proposed.

\paragraph{Spatially-varying BRDFs (SVBRDF):} For efficient SVBRDF capture, several methods based on multiplexing-based, also known as illumination-based, acquisition systems \cite{kang2018efficient, kang2019learning, ma2021free, ma2023opensvbrdf, tunwattanapong2013acquiring} have been proposed. A common approach has been to optimize the lighting patterns for efficient acquisition, followed by a BRDF fitting to an analytic model. Recent works have also leveraged deep learning architectures to learn a mapping from images to texture maps of analytic SVBRDF parameters~\cite{guo2021highlight, hui2017reflectance, deschaintre2018single, deschaintre2019flexible, martin2022materia, zhou2021adversarial,gao2019deep}. 

In contrast to those works, our focus is the reconstruction of spatially uniform BRDF that can accurately represent arbitrary complex materials. The works on spatially-varying BRDFs and efficient capture could be considered orthogonal to ours, and those methods could be potentially combined with ours. 


\subsection{Hypernetworks and GNFs}
The capacity of hypernetworks to dynamically output neural network weights, which allows models to adjust to input conditions, has drawn attention. Its promise in a variety of computer vision tasks, including dynamic network adaptation and generating neural implicit fields, has been demonstrated by recent works, such as HyperGAN \cite{ratzlaff2019hypergan} and Hyperdiffusion \cite{erkocc2023hyperdiffusion}.
The concept of a generalizable neural field has also been extensively applied to the reconstruction of neural radiance fields~\cite{wang2022attention, yang2023contranerf}, but not sufficiently studied in other domains.
These research efforts serve as our source of inspiration as we apply hypernetworks and GNFs to BRDF estimation, improving the model's adaptability to various material appearances.

