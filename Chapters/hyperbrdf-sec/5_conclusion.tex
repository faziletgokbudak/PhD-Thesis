\section{Summary}\label{sec:conc}

Photorealistic rendering can be achieved through \gls{PBR}  systems that provide a more precise portrayal of light-material interactions. A \gls{PBR}  system requires an accurate representation of \gls{BRDF}s, which can be accomplished either through analytic models that quantify \gls{BRDF}s using a few parameters or through real-world measurements captured using professional setups. While real-world \gls{BRDF} measurements offer highly accurate material appearance representations, they demand extensive capture times and specialised equipment. This chapter introduces a hypernetwork model for neural \gls{BRDF} representation, designed for point-based \gls{BRDF} acquisition setups. By representing discrete \gls{BRDF} values as a continuous generalisable neural field, HyperBRDF achieves (1) accurate reconstructions of unseen materials from sparse sample sets and (2) compression of dense \gls{BRDF} values into compact latent representations. With a set encoder enabling arbitrary input samples and a neural field (hyponet) providing nonlinear interpolation, the hypernetwork model outperforms baselines in capturing \gls{BRDF}s of test materials. Additionally, I demonstrated the model's compression capabilities on the entire MERL dataset, yielding consistent results across various materials. Integrated with the \gls{PBR} systems and capture setups, HyperBRDF has the potential to pave the way for the immersive realism that virtual world creators highly anticipate.

\textbf{Acknowledgements.} I would like to thank all the co-authors of the associated paper for their invaluable help. In this chapter, I have included figures contributed by Alejandro Sztrajman (main figure, GGX, and iBRDF results) and Chenliang Zhou (t-SNE figure) to ensure the completeness of the narrative.