\section{Summary}\label{sec:conc}

I presented a hypernetwork model for neural \gls{BRDF} representation, which is suitable for point-based \gls{BRDF} acquisition setups. Representing the discrete \gls{BRDF} values of a material as a continuous generalisable neural field, HyperBRDF offers (1) accurate reconstructions of unseen materials from a sparse set of samples and (2) compression of the highly dense \gls{BRDF} values into compact latent representations. 

Thanks to the set encoder that enables an arbitrary number of input samples and the neural field, hyponet, that offers the nonlinear built-in interpolation, I showed that the hypernetwork model can capture the \gls{BRDF}s of test materials better than the baselines. I also illustrated our model's compression capacity on the entire MERL dataset, showing consistent results across varying materials. 

Lastly, I would like to thank all the co-authors of the associated paper for their invaluable help and guidance. In this chapter, I have included some of the figures obtained with the help of Alejandro Sztrajman (main figure, GGX, and iBRDF results) and Chenliang Zhou (t-SNE figure) to ensure the completeness of the narrative.